\documentclass{article}
\title{Roll: Delivering the safety hexfecta}
\author{Isaac Grosof
\and
R. A.
\and
Sol Boucher\footnote{
The third author would have been the second author if he had written
a single \emph{word} of the paper.
At least the second author contributed actual \emph{content}.
}
}
\begin{document}
\maketitle
\begin{abstract}
Roll is a programming language pursuing the safety hexfecta.
We aim not only to achieve those safety guarantees,
but also to document those guarantees
in a fashion that transcends barriers of programming background.
\end{abstract}

Roll is a novel programming language aiming to achieve best-in-class safety,
without sacrificing the programming conviences that users expect from a
21st-century programming language.

Furthermore, we go out of our way to document our guarantees as clearly as possible.
We believe it is essential to communicate
the advantages of Roll to a wide audience,
including non-programmers of a wide range of ages.

\section{Safety hexfecta}

Far more important than what Roll can do
is what Roll can't do: Roll's safety guarantees.
Roll can best be understood through the lens of
the \emph{safety hexfexta}:

\begin{itemize}
\item Double-free pointer safety
\item Null pointer safety
\item Infinite loop and uninhabited type safety
\item Cross-function exception safety
\item Segmentation fault safety
\item Invalid representation safety
\end{itemize}
We go through each of these guarantees in turn, and in doing so explore the Roll language.
\subsection{Double-free pointer safety}
\subsection{Null pointer safety}
\subsection{Infinite loop and unihabited type safety}
\subsection{Cross-function exception safety}
\subsection{Segmentation fault safety}
\subsection{Invalid representation safety}

\section{Documentation}
\subsection{Technical jargon Glossary}
\subsection{Mid-level Documentation}
\subsection{General programming Glossary}
\subsection{Wide-audience Documentation}





\end{document}
